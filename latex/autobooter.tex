\documentclass[10pt,a4paper]{article}
\pagestyle{plain}
\textwidth 16cm
\textheight 21cm
\oddsidemargin -0.5cm
\topmargin 0cm

\title{AutoBooter}
\author{C. J. Mottram}
\date{}
\begin{document}
\pagenumbering{arabic}
\thispagestyle{empty}
\maketitle
\begin{abstract}
The document overviews the AutoBooter utility, which is used to spawn
various servers on deployment machines.
\end{abstract}

\centerline{\Large History}
\begin{center}
\begin{tabular}{|l|l|l|p{15em}|}
\hline
{\bf Version} & {\bf Author} & {\bf Date} & {\bf Notes} \\
\hline
0.1 &              C. J. Mottram & 08/08/00 & First draft \\
\hline
\end{tabular}
\end{center}

\newpage
\tableofcontents
\listoffigures
\listoftables
\newpage

\section{Introduction}
The AutoBooter is a utility process, written in Java. It is designed to be run during the startup process of the
machine(s) it is running on, and is meant to keep running until the machine is shut down again. Only one AutoBooter
process is meant to run on a machine.

The AutoBooter is responsible for spawning one or more control process that need to run on the machine. The directory,
name and command line of each process is specified in a configuration file. Each process is started in a separate 
thread. The processes are monitored for completion. Depending on the return value of the spawned processes,
they may be re-spawned, or if an engineering return value is returned, engineering mode is enabled, when the
process leaves AutoBooter's control (AutoBooter stops re-spawning it). If a process keeps re-spawning too often
(it keeps crashing or some hardware fault occurs) AutoBooter will stop re-spawning the process and write an
error message to a specified log.

\section{Execution Model}
The AutoBooter is started from an initialisation script, normally put in the network run-levels of the 
{\em init} daemon ({\em /etc/rc*.d} files). It does the following:
\begin{itemize}
\item Parses the arguments passed to the server. These can be used to set things like a non-default configuration file.
\item Loads the configuration file.
\item Goes through the list of processes to spawn in the configuration file. It starts a process thread for
	each process.
\end{itemize}

\subsection{Per-Process execution}
\label{sec:perprocessexecution}
There is a process thread spawned for each process the AutoBooter must control. 
We want it to control the process in the following manner:
\begin{itemize}
\item It must keep re-starting the process until a termination condition is met.
\item If the process terminates with an engineering status, then AutoBooter should stop re-starting
	the process and terminate this thread. This process then leaves control of the AutoBooter,
	and becomes controlled by the engineer.
\item If the process keeps terminating with an error condition, more than a certain number of times in a given
	time period, then the AutoBooter should give up trying to re-spawn the process (the process is failing
	to start because of some error).
\end{itemize}
This is achieved using the following pseudo-code:
\begin{itemize}
\item Sets up the command line strings and other variables used. This includes initialising the {\em retry start time}
	to the current time, and the {\em retry index} to zero.
\item Enters a loop, which is terminated when a termination condition is met:
     \begin{itemize}
     \item Executes the command line.
     \item Starts two stream threads, connected to the spawned process's output and error streams. These streams
	are re-directed to AutoBooter's output and error stream. These threads are needed to stop the
	spawned processes blocking on output/error messages.
     \item Waits for the spawned process to terminate, by blocking so AutoBooter takes up no processor time.
     \item If the process's return value was not a re-spawn status (e.g. the process returned a non-zero 
	return value - an error occurred) the {\em retry index} is incremented.
     \item If the {\em process end time} minus the {\em retry start time} is longer than the {\em retry time} 
	(an AutoBooter default), the {\em retry index} and {\em retry start time} are reset 
	(to zero and the current time).
     \item If the {\em retry index} is greater than the {\em retry count}, the loop terminates 
	(the process has returned too many errors is too little a time, i.e. it is failing to start correctly).
     \item If the processes return status is the {\em engineering status} (an AutoBooter default), then the 
	process has been killed in such a way that the user wants it go into engineering mode, so the loop
	is terminated.
     \end{itemize}
\item The thread then terminates.
\end{itemize}

\section{Class Hierarchy}
The following classes are used to implement the above behavior.

\subsection{AutoBooter}
This is the main start class of the AutoBooter process. The main method performs the following:
\begin{itemize}
\item Calls the {\em parseArgs} method to parse command line arguments (potentially affecting the
	configuration file used).
\item Calls the {\em init} method.
\item Calls the {\em run} method.
\end{itemize}

\subsubsection{{\em init} method}
This method creates a {\em properties} object, containing the contents of the configuration filename.
If the configuration filename cannot be found or a file error occurs, this is thrown back to
main to cause AutoBooter to terminate with an error.

\subsubsection{{\em run} method}
This looks through the loaded properties, looking for processes to start. For each process, it spawns a
{\em AutoBooterProcessThread} to deal with that process. The process's directory, command line and
retry count are all set from the loaded properties. A reference back to this class is passed 
for the process thread to request data from this class. This class implements {\em AutoBooterProcessParentInterface}
to abstract the process thread/parent thread interface away from this specific class.

Note the {\em run} method terminates as soon as all the threads are spawned and started, but because 
{\em main} does not contain a {\em System.exit} and the process threads are spawned AutoBooter does not terminate.

There is also a series of routines in this class to get data values from the properties such as
the {\em re-spawn status}, {\em engineering status} and {\em retry time}. These are retrieved from
the loaded properties.

\subsection{AutoBooterProcessParentInterface}
This is the interface between the main AutoBooter class and the per-process thread. The AutoBooter 
class implements this interface. It contains the following method declarations:
\begin{itemize}
\item {\em getReSpawnStatus}. Routine to return the status the process an instance of 
	AutoBooterProcessThread is controlling
	 will return when it needs re-spawning.
\item {\em getEngineeringStatus}. Routine to return the status the process an instance of AutoBooterProcessThread 
	is controlling will return when it wants to go into engineering mode. 
	This means the AutoBooterProcessThread for this process will terminate, and not re-start the process.
\item {\em getRetryTime}. Routine to return the retry time. If the instance of AutoBooterProcessThread re-spawns the
	process it is controlling more than {\em retry count} times in this time, the thread assumes the
	the process is continually failing and stops trying to re-start it.
\end{itemize}

\subsection{AutoBooterProcessThread}
This class extends {\em java.lang.Thread}, and implements the pseudo-code mentioned in 
Section \ref{sec:perprocessexecution}. Note the process is not run directly, it is executed via a shell command:
\verb'/bin/sh -c ( cd <directory> ; <command line> )' where \verb'<'directory\verb'>' and 
\verb'<'command line\verb'>' are read from the properties file. This is done because Java has no way of
directly specifying the directory to start a process in.

\subsection{AutoBooterStreamThread}
This class extends {\em java.lang.Thread}. One instance of this class is spawned for each stream that needs
to be read by the program. Each process has an output and error stream that needs to be read.
The thread waits for input on the stream, and re-broadcasts it to the specified stream.

\section{Directory Structure}
The AutoBooter process follows the standard Liverpool Telescope directory structure \cite{bib:ltplan}.
A general overview is shown in Figure \ref{fig:dirtree}.

\setlength{\unitlength}{1in}
\begin{figure}[!h]
	\begin{center}
		\begin{picture}(7.0,4.5)(0.0,0.0)
			\put(0,0){\special{psfile=autobooter_dirtree.eps   hscale=50 vscale=50}}
		\end{picture}
	\end{center}
	\caption{\em AutoBooter directory tree.}
	\label{fig:dirtree}
\end{figure}

\subsection{src/autobooter}
This is the top level source directory for the AutoBooter package. In here are the java and latex 
sub-directories. The {\em Makefile} is the top level makefile for this package, and calls the 
{\em Makefile}s in the java and latex sub-directories. There is also a {\em Makefile.common}
file in this directory, which contains some common definitions of directories used
by all the makefiles in this package.

\subsubsection{src/autobooter/java}
This directory contains the java source files for AutoBooter. The {\em Makefile} builds the
java sources and puts them in the bin directory tree. 

The {\em RCS} sub-directory is used 
in combination with the {\em rcs} program for version control of the source code.

The {\em *.autobooter.properties} files are the configuration files for the AutoBooter.
All the configuration files are copied to the {\em bin} sub-tree, and then the relevant 
file is soft linked as {\em autobooter.properties}, which is the default filename AutoBooter
uses to get it configuration file.

The {\em S55autobooter} file is a startup script that is also copied to the {\em bin}
sub-tree, so that it gets released to deployment machines. It should be copied into the
relevant {\em /etc/rc*.d} directories, to automatically start the AutoBooter process
then the machine is turned on.

The {\em *.autobooter.properties} and {\em S55autobooter} files should be edited in this directory
and {\em make} called to copy them to the {\em bin} sub-tree to be released, so that a copy of the
latest source files is kept.

\subsubsection{src/autobooter/latex}
This contains the sources for the \LaTeX documentation of this sub-system. This includes the
sources for this document you are reading. There is a {\em Makefile} in this directory that builds the
sources and copies them to the {\em /public\_html/latex} sub-directory, so that the web-server picks them up.

\subsection{bin/autobooter}
This directory is the root for all the binaries generated by this package.

\subsubsection{bin/autobooter/java}
This contains the class files generated from the Java sources. The AutoBooter is run from this directory.

The released version of the AutoBooter startup script ({\em S55autobooter}) is also placed here. On a 
deployment machine this should be copied into {\em /etc/rc*.d/} directories to start and stop the AutoBooter
program. 

All the {\em *.autobooter.properties} files are also copied from the source directory to this directory.
The relevant file should be soft linked to {\em autobooter.properties}, the default configuration filename
AutoBooter looks for. This can be done using a command like: 
\verb'ln -s ${HOST}.autobooter.properties autobooter.properties'. The configuration filename AutoBooter uses
can be overridden in the startup script by the {\em -config} option (see Section \ref{sec:commandlinearguments}).

\subsection{public\_html/autobooter}
This is the root directory for all documentation generated by the AutoBooter sub-system. The {\em index.html}
file points to the \LaTeX and Javadoc documentation.

\subsection{public\_html/autobooter/javadocs}
This directory contains all the documentation generated by Javadoc from the Java sources in the source directory.

\subsection{public\_html/autobooter/latex}
This directory contains all the documentation generated from the \LaTeX sources. This is the documentation you are now
reading.

\section{Command Line Arguments}
\label{sec:commandlinearguments}
The following output is produced from \verb'java AutoBooter -help':
\begin{verbatim}
AutoBooter Help:
AutoBooter starts a series of processes specified in a configuration file.
It re-spawns them when they terminate.
Arguments are:
        -[co]|[config] <filename> - Location of the configuration properties file.
        -h[elp] - show this message.
\end{verbatim}

Autobooter writes various messages to it's standard output and standard error streams during the course of operation.
The spawned process's standard output and standard error streams are re-directed to the AutoBooter's equivalent
streams, although the command line for each spawned process can include a re-direct to file. 
It is therefore a good idea to redirect AutoBooter's standard output and error streams to file
to log the operations the AutoBooter performs. The startup script does this.

\section{Configuration file}
The configuration file is a standard Java property file. The keywords AutoBooter uses is documented in
Table \ref{tab:configkeyword}. Note some keywords are per-process, so the keywords has a 
{\em \verb'<'number\verb'>'} tagged
on the end to give the keywords for that process number. The process numbers start at zero, and the 
AutoBooter will find and start threads for numbers from zero to the last non-null value for
{\em autobooter.process.name.\verb'<'number\verb'>'}.

\begin{table}
\begin{center}
\begin{tabular}{|l|l|p{12em}|l|p{17em}|}
\hline
{\bf Keyword} & {\bf Type} & {\bf Notes} \\ \hline
autobooter.process.status.engineering   & Integer & The kill value+128 to go into engineering mode.\\ \hline
autobooter.process.retry\_time		& Integer (milliseconds) & The {\em retry time} for all spawned processes.\\ \hline
autobooter.process.name.\verb'<'number\verb'>'	& String & The name of process \verb'<'number\verb'>' to spawn.\\ \hline
autobooter.process.directory.\verb'<'number\verb'>'	& String & The directory to change to before running the command line for process \verb'<'number\verb'>'.\\ \hline
autobooter.process.command\_line.\verb'<'number\verb'>'& String & The command line to start process \verb'<'number\verb'>'.\\ \hline
autobooter.process.retry\_count.\verb'<'number\verb'>'& Integer & The number of times to try to re-spawn process \verb'<'number\verb'>' it it returns a failure code e.g. {\em retry\_count}.\\ \hline
\end{tabular}
\end{center}
\caption{\em Configuration file keywords.}
\label{tab:configkeyword} 
\end{table}

An example configuration file is given below:
\begin{verbatim}
#
# AutoBooter configuration file.
#

# When exiting, spawned processes return an exit value.
# This exit value is interpreted by the AutoBooter.
autobooter.process.status.respawn	=0
# 15 is kill -SIGTERM = 143
# 1 is hangup -SIGHUP = 129
autobooter.process.status.engineering	=129
autobooter.process.retry_time		=300000

#Processes

# Ccs
autobooter.process.name.0		=Ccs
autobooter.process.directory.0		=/space/home/dev/bin/ccs/java
autobooter.process.command_line.0	=/usr/bin/nice -5 java Ccs -l 3 1> output.txt 2>&1
autobooter.process.retry_count.0	=16

# DpRt
autobooter.process.name.1		=DpRt
autobooter.process.directory.1		=/space/home/dev/bin/dprt/java
autobooter.process.command_line.1	=java DpRt -l 3 1> output.txt 2>&1
autobooter.process.retry_count.1	=16

# ccsd
autobooter.process.name.2		=ccsd
autobooter.process.directory.2		=/space/home/dev/bin/ccs/util
autobooter.process.command_line.2	=./ccsd 1> output.txt 2>&1
autobooter.process.retry_count.2	=8
\end{verbatim}

Note the {\em autobooter.process.status.engineering} property is set to 129 in this example. When a (Java)
process is killed it returns the signal number plus 128. This means to put a spawned process into 
engineering mode it should be killed with signal one, in this case SIGHUP. This is done as follows:
\begin{verbatim}
kill -SIGHUP <process id>
\end{verbatim}
This will tell AutoBooter not to respawn the process with the specified id.

\section{Suggested Improvements}
\begin{itemize}
\item When the AutoBooter is told to stop (it is killed, probably when changing run-levels),
	it does not currently kill it's spawned processes.
\item If the AutoBooter crashes, it is not currently re-spawned. If it was re-spawned in some way
	(as per a daemon startup script on Linux) it would need a mechanism for finding processes
	spawned by the last AutoBooter invocation and controlling them in some way (i.e. killing them). 
\item If one of AutoBooter's processes is put into engineering mode (so a problem can be fixed with it,
	for example) there is no way to tell AutoBooter to re-start a new thread for that process and
	re-start monitoring it (come out of engineering mdoe). The machine has to be re-booted for
	this to happen at the moment.
\end{itemize}

\begin{thebibliography}{99}
\addcontentsline{toc}{section}{Bibliography}
\bibitem{bib:ltplan}{\bf LT Software Development at the ARI}
I.A. Steele, C.J. Mottram. {\em http://ltccd1.livjm.ac.uk/\verb'~'dev/latex/plan.ps}
\end{thebibliography}

\end{document}
